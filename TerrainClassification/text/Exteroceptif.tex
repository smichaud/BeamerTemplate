\section{Classement extéroceptif}
    
\subsection{Classement extéroceptif}
    \begin{frame}
        \frametitle{Classement extéroceptif}
        \begin{itemize}
        \item Capteur utilisé : appareil stéréoscopique
            \item Données utilisées :            
            \begin{itemize}
                \item couleur en \emph{teinte saturation valeur} (HSV) ;
                \item texture visuelle (mesure des variations d'intensité locale) ;
                \item géométrie du terrain.
            \end{itemize}
            \item Trois classificateurs SVM distincts.
            \item Fusion par classificateur bayésien naif.
            \item Discrétisation en cellules (20 cm x 20 cm).
        \end{itemize}
    \end{frame}
    
\subsection{Validation expérimentale}
    \begin{frame}
        \frametitle{Validation expérimentale}
        \begin{itemize}
            \item Six ensembles de données (trajets parcourus).                        
            \item Trajets d'au moins 10 mètres.
            \item Deux ou trois types de surface.
            \item Tous les chemins étaient distincts.
            \item Récolté au cours de 3 jours.
            \item Grande variété de luminosité.
        \end{itemize}
    \end{frame}
    
\subsection{Résultats de validation}
    \begin{frame}
        \frametitle{Résultats de validation}
        Évaluation du rapport entre les VP et les FP.
        \begin{center}              
            \begin{tabular}{|lcc|}
                \hline
                Classe & VP (\%) &  FP (\%)\\
                \hline
                Roche & > 96 & < 3 \\
                Herbes & $\approx 50$ & < 0,1 \\
                Sable & > 95 & 0\\
                \hline           
            \end{tabular}                  
        \end{center}
        Au total 95,1\% des terrains étaient bien identifiés. 
    \end{frame}

            
